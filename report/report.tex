\documentclass[11pt]{article}

\usepackage{preprint}
\usepackage{amsmath, amsthm, amssymb, amsfonts}
\usepackage{resizegather}
\usepackage[numbers,square]{natbib}
\usepackage[utf8]{inputenc}
\usepackage[T1]{fontenc}
\usepackage{xcolor}
\usepackage[colorlinks = true,
    linkcolor = purple,
    urlcolor  = blue,
    citecolor = cyan,
    anchorcolor = black]{hyperref}
\usepackage{booktabs}
\usepackage{multirow}
\usepackage{nicefrac}
\usepackage{microtype}
\usepackage{lineno}
\usepackage{float}
\usepackage[shortlabels]{enumitem}
\usepackage{float}
\usepackage{subfloat}
\usepackage{caption}
\usepackage{subcaption}
\usepackage{amssymb}
\usepackage{bbold}
\usepackage{stmaryrd}
\usepackage{graphicx}
\usepackage{hyperref}
\usepackage{titlesec}
\usepackage{authblk}
\usepackage{graphics}
\usepackage[font=small,labelfont=bf]{caption}

\setlength{\belowcaptionskip}{-10pt}
\newcommand{\specialcell}[2][c]{%
  \begin{tabular}[#1]{@{}c@{}}#2\end{tabular}}


\bibliographystyle{unsrtnat}
\setlist[enumerate,1]{leftmargin=2em}
\titlespacing\section{0pt}{0.1ex plus 0.2ex minus 0.1ex}{0.1ex plus 0.2ex minus 0.1ex}
\titlespacing\subsection{0pt}{0.1ex plus 0.2ex minus 0.1ex}{0.1ex plus 0.2ex minus 0.1ex}

\renewcommand*{\Authfont}{\bfseries}

\newcommand{\R}{\mathbb{R}}
\newcommand{\N}{\mathbb{N}}
\DeclareMathOperator*{\argmin}{arg\,min}
\DeclareMathOperator*{\argmax}{arg\,max}
\DeclareMathOperator*{\minimize}{minimize}
\DeclareMathOperator*{\Argmin}{\text{Argmin}}

\title{Generative modeling project : Neural Optimal Transport}
\author[1]{Paul Barbier}
\author[1]{Bastien Le Chenadec}
\affil[1]{École des Ponts ParisTech, Master MVA}

\begin{document}

\maketitle

\begin{contribstatement}
\end{contribstatement}

\section{Introduction}

Optimal Transport (OT) is a mathematical framework that aims to find the most efficient way to transport a distribution of mass to another. This framework has been used extensively in the context of generative models, for instance as a loss function in the training of Generative Adversarial Networks (GANs) or by learning a mapping between two distributions. In this project, we aim to study the paper "Neural Optimal Transport" (Korotin, 2023) \cite{korotin-2022} which introduces an algorithm to train a neural network to learn the optimal transport between two distributions.

\section{Background on Optimal Transport}

Let $\mu$ and $\nu$ be two probability distributions on $\mathcal{X}$ and $\mathcal{Y}$ respectively (typically $\mathcal{X}, \mathcal{Y}=\R^n,\R^m$). To give a meaning to "efficiently" transporting mass, we need to define a cost function $c:\mathcal{X}\times\mathcal{Y}\to\R$ that quantifies the cost of transporting a unit of mass in $\mathcal{X}$ to one in $\mathcal{Y}$. The (Monge) optimal transport problem consists in finding a \textbf{transport map} $T^*:\mathcal{X}\to \mathcal{Y}$ such that :
\begin{equation}
    T^* \in \Argmin_{T\#\mu=\nu} \int_{\mathcal{X}} c(x,T(x))d\mu(x)
\end{equation}
where $T\#\mu$ is the pushforward distribution of $\mu$ by $T$, defined by $(T\#\mu)(A)=\mu(T^{-1}(A))$ for any measurable set $A\subset\mathcal{Y}$. This formulation calls for a deterministic mapping from $\mathcal{X}$ to $\mathcal{Y}$, which is not always desirable or feasible under general assumptions. Kantorovich introduced  a more general OT problem that aims at finding a \textbf{transport plan} $\pi^*\in \Pi(\mu,\nu)$ in the set of joint distributions on $\mathcal{X}\times\mathcal{Y}$ with marginals $\mu$ and $\nu$ such that :
\begin{equation}
    \pi^* \in \Argmin_{\pi\in\Pi(\mu,\nu)} \int_{\mathcal{X}\times\mathcal{Y}} c(x,y)d\pi(x,y)
\end{equation}
In general the solution to the Kantorovich problem is stochastic, but in some cases it may be deterministic in which case it is also a solution to the Monge problem. Following this idea of stochasticity in the solution, weak OT was introduced as a relaxation of the Kantorovich problem, where the cost function is of the form $C: \mathcal{X}\times \mathcal{P}(\mathcal{Y})\to\R$. In this case the weak OT problem writes :
\begin{equation}
    \pi^* \in \Argmin_{\pi\in\Pi(\mu,\nu)} \int_{\mathcal{X}} C(x,\pi(\cdot|x))d\pi(x)
\end{equation}
where $\pi(\cdot|x)$ is the conditional distribution of $\pi$ given $x$ and $d\pi(x)$ is the marginal distribution of $\pi$ on $\mathcal{X}$.

Building on this framework, (Korotin, 2023) \cite{korotin-2022} introduce \textbf{stochastic maps} $T:\mathcal{X}\times \mathcal{Z}\to \mathcal{Y}$ where $\mathcal{Z}$ is a latent space corresponding to the randomness in the transport. They show that the weak optimal transport problem can be reformulated and solved by a SGAD algorithm. This approach is particularly interesting in the context of generative modeling, as it allows to learn a stochastic mapping between two distributions.

\section{Conclusion}

\newpage
\bibliography{bibliography}

\newpage
\appendix

\begin{center}
    {\Large \bfseries \scshape Appendix} \\
\end{center}

\section{Figures}

\end{document}
